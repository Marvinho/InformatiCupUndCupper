\chapter{\iflanguage{english}{Methods}{Methoden}}
\label{cha:methods}

\section{Entscheidung über die zu verwendenden Methoden}

Auch wenn es zur Generierung von Irrbildern inzwischen neuere Ansätze als die Vorgestellten gibt, haben diese in der Regel einen entscheidenden Nachteil: den Zeitaufwand. Nicht nur, dass das Experimentieren mit und das Trainieren von einem oder sogar mehreren Neuronalen Netzen - sehr interessante Lösungen wie das AdvGanNet \cite{conf/ijcai/XiaoLZHLS18} verwenden drei zusammenarbeitende Netze - sehr viel Zeit kostet und dadurch schnell den Zeitrahmen des Cups erschöpfen können, auch die technische Vorgabe des Anfragelimits an die Blackbox ist ein Faktor. Um zum Beispiel unter Verwendung einer von Hinton et al. \cite{journals/corr/HintonVD15} vorgestellten Destillation der Blackbox ein exakteres Ersatznetz zu erstellen, lässt die Limitierung auf 60 Anfrage pro Minute seitens der Blackbox dieses Vorhaben schnell unrealistisch erscheinen.

Die Verwendung der im Kontext \textit{klassischen} Methode des iterativen FGSM bietet qualitativ hochwertige Ergebnisse und lässt dabei Raum für Verbesserungen an anderer Stelle, wie eine erleichterte Bedienung über die Bereitstellung einer funktionalen Benutzeroberfläche.

\section{Softwarearchitektur}

Das Zentrum der Softwarearchitektur bildet das Nutzerinterface, das durch das Skript \textit{gui.py} generiert wird. Dieses bietet die beiden Hauptfunktionalitäten:
Das Generieren bzw. Auswählen eines Basisbilds und das darauffolgende Generieren eines Irrbilds basierend auf dem gewählten Basisbild.
Alle Funktionen bezüglich des Basisbilds, wie das Generieren eines zufälligen oder einfarbigen Basisbilds oder das Auswählen und Kopieren eines nutzergewählten Basisbilds sind im Skript \textit{generateimage.py} realisiert. 
Die Funktionen zur Generierung der Irrbilder sind im Skript \textit{generateadv.py} umgesetzt. Bei der Generierung der Irrbilder wird außerdem ein vorher trainiertes Ersatznetz verwendet, das im Skript modelcnn.py umgesetzt ist und das die trainierten Gewichte aus der Datei \textit{saved\_model\_state\_CNN\_final.pth} lädt.

\section{Generierung der Irrbilder}

Zur Generierung der Irrbilder wurde die im Kapitel \textit{Theoretischer Hintergrund} dargestellte Iterative Methode zum Erreichen einer bestimmten Klasse realisiert. Der Nutzer hat dabei über das Nutzerinterface die Möglichkeit, für die Fehlklassifizierung auf eine bestimmte Klasse zu zielen oder Irrbilder für alle 43 Klassen generieren zu lassen. Die Parameter für die Anzahl der Iterationen, \(\epsilon \) und \(\alpha \) können dabei angegeben werden, um zum Beispiel optisch eher abstrakte, aber sehr schnell zu generierende Irrbilder über die Wahl hoher Werte für \( \epsilon \) und \( \alpha \) zu produzieren oder das Basisbild für das menschliche Auge nahezu unsichtbar zu verändern, indem niedrige Werte für \(\epsilon \) und \(\alpha \) gewählt werden. Bei der Wahl niedriger Werte für \(\epsilon \) und \(\alpha \) muss allerdings eine entsprechend höhere Anzahl an Iterationen durchgeführt werden, was selbstverständlich die Laufzeit der Generierung erhöht.

\section{Das Ersatznetz}

Wie im Kapitel \textit{Theoretischer Hintergrund} dargelegt, wird zum Angriff einer Blackbox mit der verwendeten Methode eine Ersatzlösung machinellen Lernens benötigt, die das gleiche Problemfeld bearbeitet. Hierfür wurde ein Convolutional Neural Network auf die Klassifizierung von Straßenverkehrsschildern mithilfe des GTSB-Datensets trainiert. Das Netz ist dabei wie folgt aufgebaut:
Drei \textit{convolutions} (c = 16, 32, 32; k= 5, 5, 3; s = 1), jeweils gefolgt von einem \textit{maxpooling} und einer \textit{relu-nonlinearity}. Danach zwei \textit{fully-connected-layer} mit 256 Knoten, die als \textit{output} Klassifizierungswahrscheinlichkeiten für die 43 Klassen generieren. Jeweils vor jeder \textit{fully-connected-layer} ist eine \textit{dropout-layer} eingezogen. Als \textit{loss} wird das \textit{cross-entropy-loss} verwendet.

\section{Testing}
 
Aufgrund der Übersichtlichkeit der Architektur und der Festlegung auf ein Referenzsystem konnte das Testen der Software mit einem eher \textit{klassischen} Ansatz, das heißt dynamische Tests der beiden Hauptfunktionen unter Verwendung verschiedener Basisbilder und Parameter, bewältigt werden. Um eine möglichst breite Auswahl an Basisbildern zu haben, wurden zum Test Bilder aus dem ------ Datenset verwendet.
Ebenso wurde das Funktionieren unter der Verwendung verschiedener Hardware sichergestellt (da über diese bei der Vorstellung der Referenzplattformen keine Informationen gegeben wurden). Insbesondere meint dies das Vorhandensein einer \textit{Cuda}-fähigen Grafikkarte.

\section{Wartbarkeit}

Da die verwendete Methode abgesehen vom Ersatznetz ein universelles Vorgehen zur Generierung von Irrbildern darstellt, lässt sich das System mit minimalen Anpassungen auch für andere Aufgaben verwenden. So kann es durch die Verwendung eines anderen Ersatznetzes schnell zur Generierung für Irrbilder für eine andere Blackbox oder direkt für eine Whitebox benutzt werden. 